 \documentclass[12pt,a4paper,oneside]{article}
\usepackage{amsfonts, amsmath, amssymb,latexsym,amsthm}
\usepackage[english]{babel}
\usepackage{epsfig}
\usepackage{enumerate}
\usepackage{graphicx} 
\usepackage{float}
\usepackage{subfigure}
\usepackage{cancel}
\usepackage{feynmf}
%\usepackage{subcaption}

\numberwithin{equation}{section}

\parskip=5pt
\parindent=15pt

\usepackage[left=1.7cm,top=2.7cm,right=1.7cm,bottom=2.7cm]{geometry}

\usepackage[utf8]{inputenc}
\usepackage{ dsfont }

\setcounter{page}{0}

\numberwithin{equation}{section}
\newtheorem{question}{Question}
\newtheorem{teorema}{Teorema}
\newtheorem{proposicio}{Proposici\'{o}}
\newtheorem{corollari}{Corol$\cdot$lari}
\newtheorem{lema}{Lema}

\theoremstyle{definition}
\newtheorem{remark}{Remark}
\newtheorem{problema}{Problem}
\newtheorem*{solucio}{Solution}
\newtheorem{exemple}{Exemple}
\newtheorem{exemples}{Exemples}
\newtheorem{observacio}{Observaci\'{o}}
\newtheorem{observacions}{Observacions}
\newcommand{\Q}{\mathds{Q}}
\newcommand{\A}{\alpha}
\newcommand{\C}{\mathds{C}}
\newcommand{\Z}{\mathds{Z}}
\newcommand{\R}{\mathds{R}}
\newcommand{\N}{\mathds{N}}
\newcommand{\F}{\mathds{F}}
\newcommand{\B}{\mathcal{B}}
\newcommand{\Irr}{\mbox{Irr}}
\newcommand{\gr}{\mbox{gr}}
\newcommand{\Gal}{\mbox{Gal}}

\def\qed{\hfill $\square$}

\usepackage{fancyhdr}

\pagestyle{fancy}

\lhead{Jorge Rodr\'{i}guez and Javier Beltr\'an}
\chead{}
\rhead{Planning and Aproximate Reasoning}

\cfoot{\thepage}

\bibstyle{plain}
\date{\today}

\begin{document}

% PORTADA
\begin{center}
	\textbf{ }\\[4cm]
	
	\LARGE Practical Exercise 1: Planner\\[0.5cm]
	\textbf{{\Huge The coffee server}}\\[0.5cm]
	\textbf{{\LARGE Planning and Approximate Reasoning}}\\[1.5cm]
	
	{\Large Universitat Rovira i Virgili}\\[2.3cm]
	{\LARGE \textbf{Master in Artificial Intelligence}}\\[0.5cm]
	{\LARGE $1^{st}$ Semester}\\[2.5cm]
	
	\begin{flushright}
		\textbf{\large{Authors:}\\ \normalsize{Javier Beltr\'an\\Jorge Rodr\'iguez\\}}
	\end{flushright}
	\today
\end{center}
\thispagestyle{empty} 
%\maketitle
\pagebreak

\section{Introduction to the problem}

For this first laboratory we were asked to consider the following scenario:

There is a squared building composed by 36 offices, which are located in a matrix of 6 rows and 6 columns. From each office it is possible to move (horizontally or vertically) to the adjacent offices. The building has some coffee machines in some offices that can make 1, 2 or 3 cups of coffee at one time.
The people working at the offices may ask for coffee and a robot called “Clooney” is in charge of serving the coffees required. Each office may ask for 1, 2 or 3 coffees but not more. The petitions of coffee are done all at early morning (just when work starts) so that the robot can plan the service procedure. Each petition has to be served in a single service.
The goal is to serve all the drinks to all the offices in an efficient way (minimizing the travel inside the building, in order to not disturb the people working).

We start with a initial state called $e_i$, which is compounded of a set of predicates. The initial state in particular has the following predicates: The initial position of the robot, the position of the coffee machines, and position of the coffee petitions.  The final state $e_f$ on the other hand, is compounded of the following predicates: The final robot position and the machines position (no more petition appears). The way of achieving this final state is by going to the machines, take the coffee and them going to each petition to leave the coffee.

To solve this problem we will implement a general planner which later will be use to solve the problem, for this we will use an algorithm called STRIPS. This algorithm is based on stacking the predicates needed to get to the final state and them finding the operators needed to get this predicates. The implementation of this has been done in Java, this will be explained in detail bellow.

As we mentioned before to formalise this problem we need to define a set of predicates and operators, for each operator we have series of preconditions, adds and deletes which are sets of predicates. This is shown in the Table\ref{op} all this together with the arguments needed for each operator:

\begin{table}[h!]
	\centering
	\caption{ Table with operators and its adds, deletes and preconditions}
	\label{op}
	\begin{tabular}{|c|l|l|l|}
		\hline
		Opertors    & \multicolumn{1}{c|}{Preconditions}                                                          & \multicolumn{1}{c|}{Adds}                                                             & \multicolumn{1}{c|}{Deletes}                                          \\ \hline
		Make(o,n)   & \begin{tabular}[c]{@{}l@{}}robot-location(o)\\ robot-free\\ machine(o,n)\end{tabular}       & robot-loaded(n)                                                                       & robot-free                                                            \\ \hline
		Move(o1,o2) & \begin{tabular}[c]{@{}l@{}}robot-location(o1)\\ steps(x)\end{tabular}                       & \begin{tabular}[c]{@{}l@{}}robot-location(o2)\\ steps(x+distance(o1,o2))\end{tabular} & \begin{tabular}[c]{@{}l@{}}robot-location(o1)\\ steps(x)\end{tabular} \\ \hline
		Serve(o,n)  & \begin{tabular}[c]{@{}l@{}}robot-location(o)\\ robot-loaded(n)\\ petition(o,n)\end{tabular} & \begin{tabular}[c]{@{}l@{}}served(o)\\ robot-free\end{tabular}                        & \begin{tabular}[c]{@{}l@{}}petition(o,n)\\ robot-loaded\end{tabular}  \\ \hline
	\end{tabular}
\end{table}
%%%%%%% Table with the operators, precondtions, adds ad deletes

The predicates are easier to define, the following list show the predicates considered for this problem:

\begin{itemize}
	\item \textbf{Robot-location(o)}: the robot is in office o.
	
	\item \textbf{Robot-free}: the robot has no cup of coffee.
	
	\item \textbf{Robot-loaded(n)}: the robot has n cups of coffee.
	
	\item \textbf{Petition(o,n)}: office o wants n cups of coffee.
	
	\item \textbf{Served(o)}: office o has been served, no more coffee is needed in this office.
		
	\item \textbf{Machine(o,n)}: there is a coffee machine in office o that produces n cups of coffee each time (with n equal to 1, 2 or 3).
	
	\item \textbf{Steps(x)}: the total distance travelled by the robot (calculated with Manhattan distance).
		
\end{itemize}

There are some consideration that can be taken so the problem is simpler:

\begin{itemize}
	\item The robot only makes coffee for a single petition each time. That is, the robot cannot make 3 cups of coffee to serve two different offices. First, will make coffee of one office and serve it, and after will go to the same or another coffee machine to serve the second office.
	
	\item If a petition is of n cups, the robot will make coffee only with a unique machine of capacity n. That is, the robot cannot make coffee with 2 or more machines and accumulate the cups to serve a unique petition.
	
\end{itemize}


\section{Analysis of the problem}

In this section we are going to  analyse the problem. First of all we notice that to the given initial state we have to  add robot-free, as to the goal state.

Taking into account the considerations to simplify the problem, this problem can be seen as an optimization problem where we want to minimize the number of steps of the robot choosing the order in which it has to visit every petition after passing for a coffee machine with the same amount of coffee as the petition. 

The search space has a total size of $|P!||M^{|P|}|$, where $|P|$ is the number of petitions and $|M|$ the number of possible machine, supposing that the robot always takes the optimal path form one point to another, so the number of possible paths is irrelevant,as the cost is always the same. To get to this result we just need to notice that we have two list of possible orders, the petitions which can't be revisited, what means a factorial; and the machines which can be revisited what means a exponential number of combinations. As we can see the number of petitions is what increases more the complexity of the problem.

For the example we have a total of 375000 possible solutions, however as we have taken the consideration that to serve a petition we need to go to a machine with the same number of coffees, the total number decreases to 1920.

In the definition of the problem the robot moves across a square grid consisting of 36 squares, the squares are called o1, o2, ..., o36. To make it easier to count the number of steps, we have considered a way to convert this to coordinates $(x,y)$.


\end{document}

